% !TeX root = ../index.tex

\section{Introduction}

Architectures of modern business applications aim to be responsive, elastic and resilient.
To that end, they commonly employ the microservice pattern and message-oriented
middleware.

For certain domain or technical reasons, the communication between microservices via the asynchronous medium may require synchronous semantics.
While some works in the professional literature \parencites{millett_patterns_2015,richardson_microservices_2019,stopford_designing_2018}\todo{add more} make reference for implementing synchronous semantics in asynchronous systems, none of them go into detail.
The academic world as well does not provide a detailed examination of this topic.

The goal of this work is to create a generic concept for implementing synchronous semantics via an asynchronous medium that fulfills the requirements of being responsive, elastic and resilient.

To that end, the work follows the design science research process outlined in \parencite{cole_being_2005}.
After section \ref{sec:foundations} establishes necessary theoretical foundations, section \ref{sec:problem} examines the problem space in greater detail.
Then, section \ref{sec:method} provides a discussion of the method chosen to approach the problem described before.
Section \ref{sec:concept} derives a concept to address the problem, which is then evaluated in section \ref{sec:evaluation}.
Lastly, section \ref{sec:conclusion} discusses the results and concludes with an outlook on further work.

\section{Foundations}\label{sec:foundations}

\subsection{Reactive Information Systems}

\subsection{Microservice Architecture}

\subsection{Message-based Communication}

\section{Distributed Business Processes In Reactive Information Systems}\label{sec:problem}



\section{Methodology}\label{sec:method}

\section{Concept}\label{sec:concept}

\section{Evaluation}\label{sec:evaluation}

\section{Conclusion}\label{sec:conclusion}
